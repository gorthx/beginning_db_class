\documentclass[20pt]{beamer}

\usepackage[orientation=landscape,size=custom,width=16,height=9,scale=0.5]{beamerposter}

\usetheme{postgres}
\setbeamertemplate{items}[circle]

\title{Introduction to Databases with PostgreSQL}
\author{Gabrielle Roth}
\institute{PDXPUG \& FreeGeek}
\date{Feb 1, 2011}

\begin{document}

\frame{
    \titlepage
}

\setbeamertemplate{footline}[postgres body]

\frame
{
    \frametitle{Outline}
    \begin{itemize}
    \item[] Intro to Databases (discussion)
    \item[] PostgreSQL (discussion)
    \item[] psql (hands-on)
    \item[] SQL (hands-on)
    \item[] ...break somewhere in here, 7:30...
    \item[] Practice db + more SQL (hands-on)
    \item[] Basic Admin + GUI Tools (discussion)
    \item[] Questions
    \end{itemize}
}

\frame[containsverbatim]
{
  \frametitle{Introductions}

  \begin{verbatim}
          __     __
         /  \~~~/  \  . o O ( Hi! )
   ,----(     oo    )
  /      \__     __/
 /|         (\  |(
^ \   /___\  /\ |
   |__|   |__|-"
  \end{verbatim}
  \tiny
  \begin{center}
  Thanks to Hayley J Wakenshaw for the Elephant
  \end{center}
}

\frame
{
    \frametitle{Databases!}
    \begin{center}
    What is it?\\
    Why do I want one?
    \end{center}
}

\frame
{
    \frametitle{Relational databases}
    \begin{itemize}
    \item[] Based on Relational Calculus
    \item[] Data is stored in "relations"
    \item[] "Normalized"
    \end{itemize}
}

\frame
{
    \frametitle{PostgreSQL}
    \begin{itemize}
    \item[-] How do you say that?
    \item[-] It's the database *server*
    \item[-] We think it's the best.
    \end{itemize}
}

\frame[containsverbatim]
{
    \frametitle{Get connected.}
    \begin{verbatim}
chmod 600 /path/to/id_pdxpug
ssh -i /path/to/id_pdxpug pdxpug@207.173.203.228
psql -U [username] -d pdxpug
    \end{verbatim}
}

\frame
{
    \frametitle{Pg command-line interface}
    \begin{itemize}
    \item[-] psql
    \item[-] - -help
    \item[-] psql -U [username] -d pdxpug
    \item[-] \textbackslash h
    \item[-] \textbackslash ?
    \item[-] other useful \textbackslash commands
    \end{itemize}
}

\frame[containsverbatim]
{
  \frametitle{This looks like a good place to talk about...}

  \begin{verbatim}
          __     __
         /  \~~~/  \  . o O ( Schemas! )
   ,----(     oo    )
  /      \__     __/
 /|         (\  |(
^ \   /___\  /\ |
   |__|   |__|-"
  \end{verbatim}
}

\frame
{
    \frametitle{SQL}
    \begin{itemize}
    \item[-] Declarative programming language
    \item[-] ...let's do "Hello World".
    \end{itemize}
}

\frame[containsverbatim]
{
    \frametitle{Hello, World.}
    \begin{verbatim}
SELECT 'Hello, World.';
    \end{verbatim}
    ...oh yeah, and there's command-completion, too.
}

\frame[containsverbatim]
{
    \frametitle{CREATE TABLE}
    \begin{verbatim}
CREATE TABLE animals
(name varchar(32) primary key,
skin varchar(32),
legs int);
    \end{verbatim}
}

\frame[containsverbatim]
{
    \frametitle{INSERT}
    \begin{verbatim}
INSERT INTO animals
VALUES ('cat', 'fur', '4');
    \end{verbatim}
}

\frame[containsverbatim]
{
    \frametitle{multi-valued INSERT}
    \begin{verbatim}
insert into animals
values
('dog', 'fur', '4'),
('bird', 'feathers', '2'),
('snake', 'scales', '0');
    \end{verbatim}
}

\frame[containsverbatim]
{
    \frametitle{SELECT}
    \begin{verbatim}
SELECT * FROM animals;

SELECT name, legs FROM animals;

SELECT * FROM animals
ORDER BY name;

SELECT count(*) FROM animals;

SELECT SUM(legs) FROM animals;
    \end{verbatim}
}

\frame[containsverbatim]
{
    \frametitle{UPDATE}
    \begin{verbatim}
UPDATE animals
SET name = 'kitty' WHERE name = 'cat';

SELECT * FROM animals;
    \end{verbatim}
}

\frame[containsverbatim]
{
    \frametitle{Transactions}
    \begin{verbatim}
BEGIN;
...your stuff...
...check your stuff with a SELECT...
ROLLBACK or COMMIT as desired
    \end{verbatim}
}

\frame[containsverbatim]
{
    \frametitle{DELETE}
    \begin{verbatim}
BEGIN;
DELETE FROM animals WHERE legs=2;
SELECT * FROM animals;
[ROLLBACK or COMMIT]
SELECT * FROM animals;
    \end{verbatim}
}

\frame
{
    \frametitle{Exercises}
    \begin{itemize}
    \item[-] add an animal of your choice
    \item[-] update its name
    \item[-] delete all animals with four legs
    \end{itemize}
}

\frame
{
    \begin{center}
    MOAR TABLEZ\\
    www.postgresqlguide.com/postgresql-sample-database.aspx
    \end{center}
}

\frame
{
    \frametitle{ERDs}
    \begin{itemize}
    \item[-] Entity-Relationship Diagrams
    \item[-] Graphical representation of the relations
    \item[-] ...and how they're connected (JOINed)
    \end{itemize}
}

\frame[containsverbatim]
{
  \begin{verbatim}
          __     __
         /  \~~~/  \  . o O ( Let's look at one. )
   ,----(     oo    )
  /      \__     __/
 /|         (\  |(
^ \   /___\  /\ |
   |__|   |__|-"
  \end{verbatim}
}

\frame[containsverbatim]
{
    \frametitle{Load 'er up!}
    \begin{verbatim}
\i /home/pdxpug/create_tables.sql
\i /home/pdxpug/populate_tables.sql
    \end{verbatim}
}

\frame
{
    \frametitle{Working with the sample db}
    \begin{itemize}
    \item[-] \textbackslash d 
    \item[-] what are these \_seq relations?
    \item[-] \textbackslash d item
    \item[-] \textbackslash s
    \item[-] \textbackslash e
    \end{itemize}
}

\frame[containsverbatim]
{
    \frametitle{DISTINCT}
    \begin{verbatim}
SELECT fname, lname FROM customer

SELECT count(lname) FROM customer;

SELECT DISTINCT lname FROM customer;

SELECT count(DISTINCT(lname)) FROM customer;
    \end{verbatim}
}

\frame
{
    \begin{itemize}
    \item[-] JOINs
    \item[-] PRIMARY keys
    \item[-] FOREIGN keys
    \end{itemize}
}

\frame
{
    \begin{center}
    We need an illustration here:\\
    INNER vs LEFT vs RIGHT vs CARTESIAN
    \end{center}
}

\frame[containsverbatim]
{
    \frametitle{JOINs}
    Note:  make the join example simpler next time.
    Compare:
    \begin{verbatim}
SELECT * FROM item;
SELECT * FROM stock;
    \end{verbatim}

    With:
    \begin{verbatim}
SELECT i.description, i.cost_price, i.sell_price, s.quantity
FROM item i
JOIN stock s ON i.item_id = s.item_id;
    \end{verbatim}

    And:
    \begin{verbatim}
SELECT i.description, i.cost_price, i.sell_price, s.quantity
FROM item i
LEFT JOIN stock s ON i.item_id = s.item_id;
    \end{verbatim}
}

\frame[containsverbatim]
{
    \begin{verbatim}
    \pset null '[null]'
    \end{verbatim}
    \begin{center}
    makes the differences more obvious.
    \end{center}
}

\frame[containsverbatim]
{
    Compare:
    \begin{verbatim}
SELECT i.description, i.cost_price, i.sell_price, s.quantity
FROM item i
RIGHT JOIN stock s ON i.item_id = s.item_id;
    \end{verbatim}

    With:
    \begin{verbatim}
SELECT i.description, i.cost_price, i.sell_price, s.quantity
FROM stock s
LEFT JOIN item i ON i.item_id = s.item_id;
    \end{verbatim}
}

\frame[containsverbatim]
{
    \frametitle{...and now the dreaded Cartesian JOIN}
    \begin{verbatim}
SELECT * FROM item;
SELECT * FROM stock;

SELECT i.description, i.cost_price, i.sell_price, s.quantity
FROM item i, stock s;

SELECT i.description, i.cost_price, i.sell_price, s.quantity
FROM item i
CROSS JOIN stock s;
    \end{verbatim}
}

\frame
{
    \frametitle{Building more complex queries}
    \begin{itemize}
    \item[-] with subSELECTs
    \item[-] using VIEWs and TEMP TABLEs
    \end{itemize}
}

\frame[containsverbatim]
{
    \frametitle{subSELECTs}
    \begin{verbatim}
SELECT date_placed FROM orderinfo
WHERE customer_id IN
    (SELECT customer_id FROM customer
    WHERE lname = 'Matthew');

SELECT date_placed FROM orderinfo
WHERE customer_id IN
(SELECT customer_id FROM customer
WHERE (fname, lname) = ('Alex','Matthew'));
    \end{verbatim}
}

\frame[containsverbatim]
{
    \frametitle{VIEWs and TEMP TABLEs}
    \begin{verbatim}
CREATE VIEW my_view AS SELECT field FROM table;
vs.
CREATE TEMP TABLE my_temp_table AS SELECT field FROM table;
    \end{verbatim}
}

\frame[containsverbatim]
{
    \frametitle{VIEWs and TEMP TABLEs}
    \begin{verbatim}
CREATE VIEW view_in_stock AS
SELECT i.item_id, i.description, s.quantity
FROM item i
LEFT JOIN stock s ON i.item_id=s.item_id;

SELECT * FROM view_in_stock;

try:
\pset null '0'
    \end{verbatim}
}

\frame
{
    \frametitle{Exercises}
    TODO Beamer:  Why isn't enumerate working?
    \begin{enumerate}
    \small
    \item[1] How many items are currently in stock?
    \item[2] In which towns do we have customers?
    \item[3] Experiment with RIGHT JOIN with the original sample queries.
    \item[4] What are the barcodes for each item?
    \item[5] What is the name + total "our cost" value of each item currently in stock?
    \item[6] What is the name + total "selling cost" value of each item currently in stock?
    \item[7] What are the total "our cost" and "sell cost" values of all items in stock?
    \item[8] How much was shipping on each order?
    \item[9] What items were ordered by people with the last name "Stones"?
    \item[10] How much was the total shipping per person?
    \end{enumerate}
}

\frame
{
    \frametitle{Basic Admin}
    \begin{itemize}
    \item[-] initdb
    \item[-] PGDATA
    \item[-] postgresql.conf
    \item[-] pg\_hba.conf
    \item[-] stop/start/restart/reload
    \end{itemize}
}

\frame
{
    \frametitle{Basic Admin}
    \begin{itemize}
    \item[-] SELECT version();
    \item[-] CREATE ROLE
    \item[-] backups and upgrades
        \begin{itemize}
        \item[-] pg\_dump, pg\_dumpall, pg\_upgrade
        \item[-] ...or replication
        \end{itemize}
    \item[-] logs 
    \end{itemize}
}

\frame
{
    \frametitle{GUI Tools}
    \begin{itemize}
    \item[-] pHpPgAdmin
    \item[-] pgAdminIII
    \end{itemize}
}

\frame[containsverbatim]
{
    \frametitle{Extras}
    \begin{itemize}
    \item[-] Natural keys vs Surrogate keys
    \item[-] Indexes
    \item[-] even more \textbackslash  commands
    \end{itemize}
    \begin{verbatim}
GRANT USAGE ON SCHEMA [schema] TO [otheruser];
GRANT SELECT ON [table] TO [otheruser];
    \end{verbatim}
}

\frame[containsverbatim]
{
    \frametitle{Book giveaway!}
    \begin{verbatim}
SELECT (random() * 100);
    \end{verbatim}
}

\frame
{
    \frametitle{Where to get help}
    \begin{itemize}
    \item[-] Postgres Community
        \begin{itemize}
        \item[-] www.postgresql.org/community/lists/
        \item[-] archives.postgresql.org/pgsql-novice/
        \item[-] \#postgresql
        \end{itemize}
    \item[-] PDXPUG
        \begin{itemize}
            \item[-] \#pdxpug
            \item[-] archives.postgresql.org/pdxpug/
        \end{itemize}
    \end{itemize}
}

\frame
{
    \frametitle{PDXPUG}
    \begin{itemize}
    \item[] Third Thurs, 7pm, FreeGeek
    \item[] pugs.postgresql.org/pdx
    \item[] @pdxpug
    \end{itemize}
}

\frame
{
    \frametitle{Reading List}
    \begin{itemize}
    \item[] \underline{Manga Guide to Databases} Takahashi, M
    \item[] \underline{Database Design for Mere Mortals} Hernandez, M J
    \item[] \underline{SQL for Smarties} Celko, J
    \item[] \underline{Introduction to Database Systems} Date, CJ
    \item[] \underline{Relational Model for Database Management} Codd, EF
    \item[] www.postgresql.org/docs
    \item[] \underline{PostgreSQL 9 Admin Cookbook} Riggs, S \& H Krosing
    \item[] \underline{PostgreSQL 9.0 High Performance} Smith, G
    \end{itemize}
}

\frame[containsverbatim]
{
  \begin{verbatim}
          __     __
         /  \~~~/  \  . o O ( Thank you! )
   ,----(     oo    )
  /      \__     __/
 /|         (\  |(
^ \   /___\  /\ |
   |__|   |__|-"
  \end{verbatim}

}

\end{document}
